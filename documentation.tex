\documentclass{article}
\usepackage{amsmath}
\usepackage{geometry}
\usepackage{graphicx}
\usepackage{hyperref}
\usepackage{listings}
\usepackage{minted} % For better code highlighting if available, otherwise listings

\geometry{a4paper, margin=1in}

\title{Quantum Dot Simulation Suite Documentation}
\author{AI Software Developer}
\date{\today}

\begin{document}
\maketitle
\tableofcontents
\newpage

\section{Introduction}
This document provides an overview of the physics and code structure for a suite of quantum dot simulators. These simulators solve the Schrödinger and Poisson equations self-consistently to model the behavior of electrons in semiconductor quantum dot devices. The suite includes tools for 1D and 2D simulations, as well as specialized simulations for pinch-off characteristics, charge stability diagrams, and Coulomb diamonds.

\section{Physics Background}
The core of the simulation lies in the self-consistent solution of the Schrödinger and Poisson equations.

\subsection{Schrödinger Equation}
The time-independent Schrödinger equation describes the quantum mechanical states of electrons in a given potential. For a single electron with effective mass $m_{\text{eff}}$ in a potential $V(\mathbf{r})$, the equation is:
\begin{equation}
    \left[ -\frac{\hbar^2}{2m_{\text{eff}}} \nabla^2 + V(\mathbf{r}) \right] \psi_i(\mathbf{r}) = E_i \psi_i(\mathbf{r})
\end{equation}
where $\hbar$ is the reduced Planck constant, $\psi_i(\mathbf{r})$ are the eigenfunctions (wavefunctions), and $E_i$ are the corresponding eigenvalues (energy levels). The total potential $V(\mathbf{r})$ is a sum of the external potential $V_{\text{ext}}(\mathbf{r})$ (due to gates) and the electrostatic potential $\phi(\mathbf{r})$ due to the electron charge itself: $V(\mathbf{r}) = V_{\text{ext}}(\mathbf{r}) - e \phi(\mathbf{r})$.

\subsection{Poisson Equation}
The Poisson equation relates the electrostatic potential $\phi(\mathbf{r})$ to the charge density $\rho(\mathbf{r})$:
\begin{equation}
    \nabla^2 \phi(\mathbf{r}) = -\frac{\rho(\mathbf{r})}{\epsilon}
\end{equation}
where $\epsilon = \epsilon_r \epsilon_0$ is the permittivity of the material ($\epsilon_r$ is the relative permittivity and $\epsilon_0$ is the vacuum permittivity). The charge density $\rho(\mathbf{r})$ is calculated from the occupied electron states:
\begin{equation}
    \rho(\mathbf{r}) = -e \sum_i f(E_i, E_F, T) |\psi_i(\mathbf{r})|^2
\end{equation}
where $e$ is the elementary charge, and $f(E_i, E_F, T)$ is the Fermi-Dirac distribution function, which gives the probability of occupation for an energy state $E_i$ at a given Fermi level $E_F$ and temperature $T$. A factor of 2 is included for spin degeneracy.

\subsection{Self-Consistent Solution}
The Schrödinger and Poisson equations are coupled: the potential in the Schrödinger equation depends on the charge density (via the Poisson equation), and the charge density in the Poisson equation depends on the wavefunctions (via the Schrödinger equation). This necessitates a self-consistent solution:
\begin{enumerate}
    \item \textbf{Initial Guess}: Start with an initial guess for the electrostatic potential $\phi(\mathbf{r})$ (e.g., $\phi(\mathbf{r}) = 0$).
    \item \textbf{Total Potential}: Calculate the total potential $V(\mathbf{r}) = V_{\text{ext}}(\mathbf{r}) - e \phi(\mathbf{r})$.
    \item \textbf{Solve Schrödinger Equation}: Solve the Schrödinger equation using $V(\mathbf{r})$ to find eigenvalues $E_i$ and eigenvectors $\psi_i(\mathbf{r})$.
    \item \textbf{Calculate Charge Density}: Calculate the new charge density $\rho_{\text{new}}(\mathbf{r})$ using the computed $E_i$ and $\psi_i(\mathbf{r})$.
    \item \textbf{Solve Poisson Equation}: Solve the Poisson equation with $\rho_{\text{new}}(\mathbf{r})$ to obtain a new electrostatic potential $\phi_{\text{new}}(\mathbf{r})$.
    \item \textbf{Check Convergence}: Compare $\phi_{\text{new}}(\mathbf{r})$ with the previous $\phi(\mathbf{r})$. If the difference is below a tolerance, the solution is converged.
    \item \textbf{Mix and Iterate}: If not converged, mix the new and old potentials (e.g., $\phi(\mathbf{r}) = (1-\alpha)\phi(\mathbf{r}) + \alpha\phi_{\text{new}}(\mathbf{r})$, where $\alpha$ is a mixing parameter) and return to step 2.
\end{enumerate}
This iterative process is repeated until convergence is achieved.

\section{Code Documentation}
The simulation suite consists of several Python scripts, each tailored for specific types of simulations. They share common physical constants, material parameters, and core solver functions.

\subsection{Common Elements}
\subsubsection{Physical Constants and Material Parameters}
All scripts define:
\begin{itemize}
    \item \textbf{Physical Constants}: $\hbar$ (hbar), $m_e$ (electron mass), $e$ (elementary charge), $\epsilon_0$ (vacuum permittivity).
    \item \textbf{Material Parameters (GaAs)}: $m_{\text{eff}}$ (effective mass, $0.067 \cdot m_e$), $\epsilon_r$ (relative permittivity, $12.9$).
\end{itemize}

\subsubsection{Simulation Grid}
A 1D or 2D numerical grid is defined:
\begin{itemize}
    \item \texttt{L} (1D) or \texttt{Lx}, \texttt{Ly} (2D): Length of the simulation domain.
    \item \texttt{N} (1D) or \texttt{Nx}, \texttt{Ny} (2D): Number of grid points.
    \item \texttt{x}, \texttt{y}: Arrays of grid point coordinates.
    \item \texttt{dx}, \texttt{dy}: Grid spacing.
\end{itemize}
For 2D simulations, \texttt{X} and \texttt{Y} meshgrids are created.

\subsubsection{Core Functions}
\begin{description}
    \item[\texttt{get\_external\_potential(X, Y, voltages)}] Calculates the external potential profile based on gate voltages. Uses Gaussian profiles for each gate's influence. The \texttt{voltages} argument is a dictionary mapping gate names (e.g., "P1", "B2") to their applied voltage values.
    \item[\texttt{solve\_schrodinger(potential)}] (1D) or \texttt{solve\_schrodinger\_2d(potential\_2d)}] Solves the time-independent Schrödinger equation using a finite difference method on a sparse matrix representation of the Hamiltonian. Returns eigenvalues and eigenvectors.
    \item[\texttt{calculate\_charge\_density(eigenvalues, eigenvectors, fermi\_level)}] (1D) or \texttt{calculate\_charge\_density\_2d(eigenvalues, eigenvectors\_2d, fermi\_level)}] Calculates the electron charge density using Fermi-Dirac statistics (or a zero-temperature approximation).
    \item[\texttt{solve\_poisson(charge\_density)}] (1D) or \texttt{solve\_poisson\_2d(charge\_density\_2d)}] Solves the Poisson equation using a finite difference method with Dirichlet boundary conditions (potential fixed at boundaries, typically to zero).
    \item[\texttt{solve\_poisson\_2d\_spectral(charge\_density\_2d)}] (2D only) An alternative Poisson solver using spectral methods (FFT), assuming periodic boundary conditions.
    \item[\texttt{self\_consistent\_solver(voltages, fermi\_level, ...)}] (1D) or \texttt{self\_consistent\_solver\_2d(voltages, fermi\_level, ...)}] Implements the self-consistent iteration loop described in Section 2.3. It takes applied gate voltages and the Fermi level as input and returns the converged total potential, charge density, eigenvalues, and eigenvectors.
\end{description}

\subsection{\texttt{simulate\_1d\_dot.py}}
This script performs a 1D Schrödinger-Poisson simulation for a quantum dot device.
\begin{itemize}
    \item \textbf{Purpose}: Simulates a 1D quantum dot, typically a double dot structure defined by plunger and barrier gates.
    \item \textbf{Key Features}:
        \begin{itemize}
            \item Defines a 1D grid.
            \item \texttt{get\_external\_potential}: Models gates using 1D Gaussian potentials.
            \item Solves 1D Schrödinger and Poisson equations.
            \item Performs self-consistent calculation.
            \item Plots potential profiles, charge density, and wavefunctions.
        \end{itemize}
\end{itemize}

\subsection{\texttt{simulate\_2d\_dot.py}}
This script performs a 2D Schrödinger-Poisson simulation.
\begin{itemize}
    \item \textbf{Purpose}: Simulates a 2D quantum dot, allowing for more realistic device geometries.
    \item \textbf{Key Features}:
        \begin{itemize}
            \item Defines a 2D grid (\texttt{Nx}, \texttt{Ny}).
            \item \texttt{get\_external\_potential}: Models gates using 2D Gaussian potentials.
            \item \texttt{solve\_schrodinger\_2d}: Solves the 2D Schrödinger equation. The Hamiltonian is constructed using a 5-point finite difference stencil.
            \item \texttt{solve\_poisson\_2d}: Solves the 2D Poisson equation using finite differences with Dirichlet boundary conditions.
            \item \texttt{solve\_poisson\_2d\_spectral}: An alternative spectral Poisson solver.
            \item \texttt{self\_consistent\_solver\_2d}: Manages the 2D self-consistent loop. Allows selection of Poisson solver type.
            \item Plots 2D contour maps of potentials, charge density, and the ground state probability density. Includes visualization of gate positions.
        \end{itemize}
\end{itemize}

\subsection{\texttt{simulate\_pinchoff.py}}
This script simulates the pinch-off characteristics of a quantum dot device by sweeping a gate voltage.
\begin{itemize}
    \item \textbf{Purpose}: To observe how the potential barrier under a gate changes as its voltage is swept, leading to channel pinch-off.
    \item \textbf{Key Features}:
        \begin{itemize}
            \item Based on the 2D simulation framework.
            \item Sweeps the voltage of a specified gate (e.g., "B2") over a defined range.
            \item For each voltage point in the sweep:
                \begin{itemize}
                    \item Runs the 2D self-consistent solver.
                    \item Estimates the minimum potential under the swept gate (barrier height).
                    \item Calculates the total number of electrons in the simulation domain.
                \end{itemize}
            \item Plots:
                \begin{itemize}
                    \item Pinch-off curve: Minimum barrier potential vs. swept gate voltage.
                    \item Number of electrons vs. swept gate voltage.
                    \item Final potential landscape for the last voltage point.
                \end{itemize}
            \item Allows selection between finite difference and spectral Poisson solvers.
        \end{itemize}
\end{itemize}

\subsection{\texttt{simulate\_charge\_stability.py}}
This script simulates charge stability diagrams by sweeping two gate voltages.
\begin{itemize}
    \item \textbf{Purpose}: To map out regions of stable electron numbers in the quantum dot system as two plunger gate voltages are varied. This reveals the characteristic honeycomb pattern of charge stability.
    \item \textbf{Key Features}:
        \begin{itemize}
            \item Based on the 2D simulation framework, often with a reduced grid size for faster computation over many voltage points.
            \item Sweeps two gate voltages (e.g., "P1", "P2") over 2D ranges.
            \item For each pair of (V\textsubscript{P1}, V\textsubscript{P2}) values:
                \begin{itemize}
                    \item Runs the 2D self-consistent solver.
                    \item Calculates the total number of electrons in the system using \texttt{calculate\_total\_electrons}.
                \end{itemize}
            \item Implements sweep strategies:
                \begin{itemize}
                    \item \texttt{row\_by\_row}: Standard raster scan.
                    \item \texttt{hilbert}: Sweeps points along a Hilbert space-filling curve, which can improve convergence by using the solution from a nearby point in parameter space as a warm start for the current point's self-consistent calculation. The \texttt{get\_hilbert\_order} function generates this sequence.
                \end{itemize}
            \item Stores the converged electrostatic potential from the previous point to use as an initial guess (warm start) for the next point, potentially speeding up convergence.
            \item Plots:
                \begin{itemize}
                    \item 2D color map of the total number of electrons as a function of the two swept gate voltages.
                    \item An additional plot with rounded electron numbers to emphasize plateaus.
                \end{itemize}
            \item Saves raw data and plots to an output directory.
        \end{itemize}
\end{itemize}

\subsection{\texttt{simulate\_coulomb\_diamonds.py}}
This script simulates Coulomb diamonds, which are related to charge stability diagrams but typically involve sweeping gate voltages and observing transitions in electron number.
\begin{itemize}
    \item \textbf{Purpose}: Similar to charge stability, this script maps out electron number as a function of two gate voltages, often used to identify Coulomb blockade regions.
    \item \textbf{Key Features}:
        \begin{itemize}
            \item Based on the 2D simulation framework.
            \item Sweeps two gate voltages (e.g., "P1", "P2").
            \item For each voltage pair, runs the self-consistent solver and calculates the total number of electrons.
            \item Plots a 2D color map of the electron number, forming the Coulomb diamond pattern.
            \item Uses a zero-temperature approximation for charge density calculation.
        \end{itemize}
\end{itemize}

\section{Numerical Methods}
\subsection{Finite Difference Method}
The Schrödinger and Poisson equations are discretized using the finite difference method. For example, the second derivative $\frac{d^2\psi}{dx^2}$ is approximated as:
\begin{equation}
    \frac{d^2\psi}{dx^2} \approx \frac{\psi_{i+1} - 2\psi_i + \psi_{i-1}}{dx^2}
\end{equation}
where $\psi_i$ is the value of $\psi$ at grid point $x_i$, and $dx$ is the grid spacing. This transforms the differential equations into a system of linear algebraic equations, which can be represented by sparse matrices.

\subsection{Eigenvalue Solvers}
The discretized Schrödinger equation becomes a matrix eigenvalue problem $H\psi = E\psi$. Sparse eigenvalue solvers (e.g., \texttt{scipy.sparse.linalg.eigsh}) are used to find the lowest energy eigenvalues and corresponding eigenvectors.

\subsection{Linear System Solvers}
The discretized Poisson equation becomes a linear system $A\phi = b$. Sparse linear solvers (e.g., \texttt{scipy.sparse.linalg.spsolve} or iterative solvers like GMRES) are used to find the potential $\phi$.

\subsection{Spectral Method (Poisson)}
The \texttt{solve\_poisson\_2d\_spectral} function uses Fast Fourier Transforms (FFTs) to solve the Poisson equation in k-space. The equation $\nabla^2 \phi = -\rho/\epsilon$ becomes $-K^2 \Phi_k = -P_k/\epsilon$ in Fourier space, where $K^2 = k_x^2 + k_y^2$, and $\Phi_k, P_k$ are the Fourier transforms of $\phi, \rho$. This method is efficient but assumes periodic boundary conditions, which may not always be appropriate for device simulation where Dirichlet conditions are often preferred.

\section{Conclusion}
This simulation suite provides a flexible framework for modeling quantum dot devices. By solving the Schrödinger and Poisson equations self-consistently, it can predict electron distributions and energy levels under various gate configurations. The specialized scripts allow for the simulation of important experimental characteristics like pinch-off, charge stability diagrams, and Coulomb diamonds, offering valuable insights into quantum dot physics.

\end{document}
